%% ----------------------------------------------------------------
%% Thesis.tex -- MAIN FILE (the one that you compile with LaTeX)
%% ---------------------------------------------------------------- 

% Set up the document
%\documentclass[a4paper, 11pt, oneside]{Thesis}  % Use the "Thesis" style, based on the ECS Thesis style by Steve Gunn
\documentclass[a4paper, 11pt,oneside]{$thesis_class$}
\graphicspath{{Figures/}}  % Location of the graphics files (set up for graphics to be in PDF format)
%
% Include any extra LaTeX packages required
%\usepackage[square, numbers, comma, sort&compress]{natbib}  % Use the "Natbib" style for the references in the Bibliography
%\usepackage{chicago}
%\usepackage{arial}
\usepackage[sort]{natbib}
\usepackage{amsmath,amssymb,amsfonts,amsthm}
\usepackage{array}
\usepackage[dvips]{color}  % Allows highlighting in text
\usepackage{epsf}
\usepackage[T1]{fontenc}
\usepackage{graphicx}
\usepackage{hyperref}
\usepackage{longtable}
\usepackage{lscape} % rotate figures / text / tables to landscape
\usepackage[version=3]{$build$/latex/mhchem}
\usepackage{multirow}
\usepackage{paralist} % lists inline in text
\usepackage{rotating}
\usepackage{sfmath}
\usepackage{subfigure}
\usepackage{textcomp}
\usepackage{txfonts}
\usepackage{url}
\usepackage{$build$/latex/vector}  % Allows "\bvec{}" and "\buvec{}" for "blackboard" style bold vectors in maths
\usepackage{$build$/latex/lstpatch}
\usepackage{verbatim}  % Needed for the "comment" environment to make LaTeX comments
\usepackage[usenames,dvipsnames,svgnames,table]{xcolor}

% used to stop "Too many unprocessed floats"
% see http://www.douglasvanbossuyt.com/2008/11/18/latex-too-many-unprocessed-floats-problem-and-solution/
\usepackage[section] {placeins}

\hypersetup{urlcolor=black, colorlinks=false}  % Colours hyperlinks in blue, but this can be distracting if there are many links.

%\usepackage[scaled = 0.92]{helvet}
%\renewcommand*\familydefault{\sfdefault}
% Uncomment the above two lines for the Helvetica font 

% Hue = 205, Value = 100, 
\definecolor{high}{RGB}{0,148,255} %S=100
\definecolor{medium}{RGB}{102,191,255}  %S=60
\definecolor{low}{RGB}{178,223,255} %S=30

\definecolor{pos}{RGB}{255,182,193} 
\definecolor{neg}{RGB}{182,193,255} 

\hyphenation{fronto-genesis}
\hyphenation{cyclo-genesis}
\hyphenation{fronto-lysis}
\hyphenation{cyclo-lysis}
\hyphenation{radio-sonde}
\hyphenation{semi-geo-stro-phic}
\hyphenation{quasi-geo-stro-phic}
\hyphenation{extra-trop-i-cal}

\renewcommand{\bibname}{References}
\newcommand{\ignore}[1]{}
%\renewcommand{\familydefault}{\sfdefault}
%% ----------------------------------------------------------------
\begin{document}
%
%% Set up the Title Page

\title{$title$}
\supervisor  {}
\examiner    {}
\degree      {}
\university  {\href{$university.url$}{$university.name$}}
\group       {\href{$group.url$}{$group.name$}}
\department  {\href{$department.url$}{$department.name$}}
\authors     {\href{$author.email$}{$author.name$}}
\addresses   {\groupname\\\deptname\\\univname}  % Do not change this here, instead these must be set in the "Thesis.cls" file, please look through it instead
\date        {\today}
\subject     {}
\keywords    {}

\maketitle


\setstretch{1.5}  % It is better to have smaller font and larger line spacing than the other way round
\fancyhead{}  % Clears all page headers and footers
\rhead{\thepage}  % Sets the right side header to show the page number
\lhead{}  % Clears the left side page header

\pagestyle{fancy}  % Use the "fancy" page style to implement the FancyHdr headers

%----------------------------------------------------------------
$prelims$
% ----------------------------------------------------------------
\mainmatter	  % Begin normal, numeric (1,2,3...) page numbering
\pagestyle{fancy}  % Return the page headers back to the "fancy" style

% Include the chapters of the thesis, as separate files
% Just uncomment the lines as you write the chapters

$body$

% ----------------------------------------------------------------
\backmatter
$postlims$
%----------------------------------------------------------------
\end{document}  % The End
